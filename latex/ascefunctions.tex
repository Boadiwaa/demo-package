%% Generated by Sphinx.
\def\sphinxdocclass{report}
\documentclass[letterpaper,10pt,english]{sphinxmanual}
\ifdefined\pdfpxdimen
   \let\sphinxpxdimen\pdfpxdimen\else\newdimen\sphinxpxdimen
\fi \sphinxpxdimen=.75bp\relax
\ifdefined\pdfimageresolution
    \pdfimageresolution= \numexpr \dimexpr1in\relax/\sphinxpxdimen\relax
\fi
%% let collapsible pdf bookmarks panel have high depth per default
\PassOptionsToPackage{bookmarksdepth=5}{hyperref}

\PassOptionsToPackage{booktabs}{sphinx}
\PassOptionsToPackage{colorrows}{sphinx}

\PassOptionsToPackage{warn}{textcomp}
\usepackage[utf8]{inputenc}
\ifdefined\DeclareUnicodeCharacter
% support both utf8 and utf8x syntaxes
  \ifdefined\DeclareUnicodeCharacterAsOptional
    \def\sphinxDUC#1{\DeclareUnicodeCharacter{"#1}}
  \else
    \let\sphinxDUC\DeclareUnicodeCharacter
  \fi
  \sphinxDUC{00A0}{\nobreakspace}
  \sphinxDUC{2500}{\sphinxunichar{2500}}
  \sphinxDUC{2502}{\sphinxunichar{2502}}
  \sphinxDUC{2514}{\sphinxunichar{2514}}
  \sphinxDUC{251C}{\sphinxunichar{251C}}
  \sphinxDUC{2572}{\textbackslash}
\fi
\usepackage{cmap}
\usepackage[T1]{fontenc}
\usepackage{amsmath,amssymb,amstext}
\usepackage{babel}



\usepackage{tgtermes}
\usepackage{tgheros}
\renewcommand{\ttdefault}{txtt}



\usepackage[Bjarne]{fncychap}
\usepackage{sphinx}

\fvset{fontsize=auto}
\usepackage{geometry}


% Include hyperref last.
\usepackage{hyperref}
% Fix anchor placement for figures with captions.
\usepackage{hypcap}% it must be loaded after hyperref.
% Set up styles of URL: it should be placed after hyperref.
\urlstyle{same}

\addto\captionsenglish{\renewcommand{\contentsname}{Contents:}}

\usepackage{sphinxmessages}
\setcounter{tocdepth}{1}



\title{ascefunctions}
\date{Oct 18, 2024}
\release{0.10}
\author{Paulina Boadiwaa Mensah}
\newcommand{\sphinxlogo}{\vbox{}}
\renewcommand{\releasename}{Release}
\makeindex
\begin{document}

\ifdefined\shorthandoff
  \ifnum\catcode`\=\string=\active\shorthandoff{=}\fi
  \ifnum\catcode`\"=\active\shorthandoff{"}\fi
\fi

\pagestyle{empty}
\sphinxmaketitle
\pagestyle{plain}
\sphinxtableofcontents
\pagestyle{normal}
\phantomsection\label{\detokenize{index::doc}}



\chapter{ascefunctions}
\label{\detokenize{index:ascefunctions}}
\sphinxAtStartPar
This package consists of 2 modules: taylors\_series and bessel\_group used for trigonometric
and Bessel functions respectively, for use on \sphinxcode{\sphinxupquote{scalar and numpy array}} objects.
The trigonometric functions were created using Taylors series. %
\begin{footnote}[1]\sphinxAtStartFootnote
\sphinxurl{https://mathworld.wolfram.com/TaylorSeries.html}
%
\end{footnote} The Bessel function was implemented based on Euler’s definition. %
\begin{footnote}[2]\sphinxAtStartFootnote
\sphinxurl{https://mathworld.wolfram.com/BesselFunctionoftheFirstKind.html}
%
\end{footnote}

\sphinxAtStartPar
For examples see {\hyperref[\detokenize{index:ascefunctions.taylors_series.trig_functions.exp}]{\sphinxcrossref{\sphinxcode{\sphinxupquote{ascefunctions.taylors\_series.trig\_functions.exp()}}}}} and {\hyperref[\detokenize{index:ascefunctions.taylors_series.trig_functions.sinh}]{\sphinxcrossref{\sphinxcode{\sphinxupquote{ascefunctions.taylors\_series.trig\_functions.sinh()}}}}} for more information.


\section{\sphinxstylestrong{Here are the functions in the taylors\_series module:}}
\label{\detokenize{index:module-ascefunctions.taylors_series}}\label{\detokenize{index:here-are-the-functions-in-the-taylors-series-module}}\index{module@\spxentry{module}!ascefunctions.taylors\_series@\spxentry{ascefunctions.taylors\_series}}\index{ascefunctions.taylors\_series@\spxentry{ascefunctions.taylors\_series}!module@\spxentry{module}}\index{module@\spxentry{module}!ascefunctions.taylors\_series.trig\_functions@\spxentry{ascefunctions.taylors\_series.trig\_functions}}\index{ascefunctions.taylors\_series.trig\_functions@\spxentry{ascefunctions.taylors\_series.trig\_functions}!module@\spxentry{module}}\index{cosh() (in module ascefunctions.taylors\_series.trig\_functions)@\spxentry{cosh()}\spxextra{in module ascefunctions.taylors\_series.trig\_functions}}\phantomsection\label{\detokenize{index:module-ascefunctions.taylors_series.trig_functions}}

\begin{fulllineitems}
\phantomsection\label{\detokenize{index:ascefunctions.taylors_series.trig_functions.cosh}}
\pysigstartsignatures
\pysiglinewithargsret
{\sphinxcode{\sphinxupquote{ascefunctions.taylors\_series.trig\_functions.}}\sphinxbfcode{\sphinxupquote{cosh}}}
{\sphinxparam{\DUrole{n}{x}}\sphinxparamcomma \sphinxparam{\DUrole{n}{terms}\DUrole{o}{=}\DUrole{default_value}{20}}}
{}
\pysigstopsignatures
\sphinxAtStartPar
Compute cosh(x) using Taylor series expansion.
\begin{quote}\begin{description}
\sphinxlineitem{Parameters}\begin{itemize}
\item {} 
\sphinxAtStartPar
\sphinxstyleliteralstrong{\sphinxupquote{np.array}} (\sphinxstyleliteralemphasis{\sphinxupquote{int}}\sphinxstyleliteralemphasis{\sphinxupquote{ or }}\sphinxstyleliteralemphasis{\sphinxupquote{float}}\sphinxstyleliteralemphasis{\sphinxupquote{ or }}\sphinxstyleliteralemphasis{\sphinxupquote{list}}\sphinxstyleliteralemphasis{\sphinxupquote{ of }}\sphinxstyleliteralemphasis{\sphinxupquote{integers or}}) \textendash{} x: The input value for cosh(x). Can be a scalar or a numpy array.

\item {} 
\sphinxAtStartPar
\sphinxstyleliteralstrong{\sphinxupquote{int}} \textendash{} \begin{description}
\sphinxlineitem{terms: Number of terms in the Taylor series to consider}
\sphinxAtStartPar
(higher = more accurate).

\end{description}


\end{itemize}

\sphinxlineitem{Returns}
\sphinxAtStartPar
Approximate value of cosh(x). Can be a scalar or a numpy array.

\sphinxlineitem{Return type}
\sphinxAtStartPar
np.float or np.array

\end{description}\end{quote}
\subsubsection*{Examples}

\begin{sphinxVerbatim}[commandchars=\\\{\}]
\PYG{g+gp}{\PYGZgt{}\PYGZgt{}\PYGZgt{} }\PYG{n}{cosh}\PYG{p}{(}\PYG{l+m+mi}{1}\PYG{p}{)}
\PYG{g+go}{array(1.54308063)}
\end{sphinxVerbatim}

\begin{sphinxVerbatim}[commandchars=\\\{\}]
\PYG{g+gp}{\PYGZgt{}\PYGZgt{}\PYGZgt{} }\PYG{n}{cosh}\PYG{p}{(}\PYG{p}{[}\PYG{l+m+mf}{1.0}\PYG{p}{,} \PYG{l+m+mf}{2.0}\PYG{p}{,} \PYG{l+m+mf}{3.0}\PYG{p}{,} \PYG{l+m+mf}{4.0}\PYG{p}{,} \PYG{l+m+mf}{5.0}\PYG{p}{]}\PYG{p}{)}
\PYG{g+go}{array([ 1.54308063,  3.76219569, 10.067662  , 27.30823284, 74.20994852])}
\end{sphinxVerbatim}

\end{fulllineitems}

\index{exp() (in module ascefunctions.taylors\_series.trig\_functions)@\spxentry{exp()}\spxextra{in module ascefunctions.taylors\_series.trig\_functions}}

\begin{fulllineitems}
\phantomsection\label{\detokenize{index:ascefunctions.taylors_series.trig_functions.exp}}
\pysigstartsignatures
\pysiglinewithargsret
{\sphinxcode{\sphinxupquote{ascefunctions.taylors\_series.trig\_functions.}}\sphinxbfcode{\sphinxupquote{exp}}}
{\sphinxparam{\DUrole{n}{x}}\sphinxparamcomma \sphinxparam{\DUrole{n}{terms}\DUrole{o}{=}\DUrole{default_value}{20}}}
{}
\pysigstopsignatures
\sphinxAtStartPar
Compute e\textasciicircum{}x using Taylor series expansion.
\begin{quote}\begin{description}
\sphinxlineitem{Parameters}
\sphinxAtStartPar
\sphinxstyleliteralstrong{\sphinxupquote{np.array}} (\sphinxstyleliteralemphasis{\sphinxupquote{int}}\sphinxstyleliteralemphasis{\sphinxupquote{ or }}\sphinxstyleliteralemphasis{\sphinxupquote{float}}\sphinxstyleliteralemphasis{\sphinxupquote{ or }}\sphinxstyleliteralemphasis{\sphinxupquote{list}}\sphinxstyleliteralemphasis{\sphinxupquote{ of }}\sphinxstyleliteralemphasis{\sphinxupquote{integers or}}) \textendash{} 
\sphinxAtStartPar
x: The exponent value for e\textasciicircum{}x. Can be a scalar or a numpy array.
terms: Number of terms in the Taylor series to consider
\begin{quote}

\sphinxAtStartPar
(higher = more accurate).
\end{quote}


\sphinxlineitem{Returns}
\sphinxAtStartPar
Approximate value of e\textasciicircum{}x. Can be a scalar or a numpy array.

\sphinxlineitem{Return type}
\sphinxAtStartPar
np.float or np.array

\end{description}\end{quote}
\subsubsection*{Examples}

\begin{sphinxVerbatim}[commandchars=\\\{\}]
\PYG{g+gp}{\PYGZgt{}\PYGZgt{}\PYGZgt{} }\PYG{n}{exp}\PYG{p}{(}\PYG{l+m+mf}{2.0}\PYG{p}{)}
\PYG{g+go}{array(7.3890561)}
\end{sphinxVerbatim}

\begin{sphinxVerbatim}[commandchars=\\\{\}]
\PYG{g+gp}{\PYGZgt{}\PYGZgt{}\PYGZgt{} }\PYG{n}{exp}\PYG{p}{(}\PYG{p}{[}\PYG{l+m+mf}{1.0}\PYG{p}{,} \PYG{l+m+mf}{2.0}\PYG{p}{,} \PYG{l+m+mf}{3.0}\PYG{p}{]}\PYG{p}{)}
\PYG{g+go}{array([ 2.71828183,  7.3890561 , 20.08553692])}
\end{sphinxVerbatim}

\end{fulllineitems}

\index{sinh() (in module ascefunctions.taylors\_series.trig\_functions)@\spxentry{sinh()}\spxextra{in module ascefunctions.taylors\_series.trig\_functions}}

\begin{fulllineitems}
\phantomsection\label{\detokenize{index:ascefunctions.taylors_series.trig_functions.sinh}}
\pysigstartsignatures
\pysiglinewithargsret
{\sphinxcode{\sphinxupquote{ascefunctions.taylors\_series.trig\_functions.}}\sphinxbfcode{\sphinxupquote{sinh}}}
{\sphinxparam{\DUrole{n}{x}}\sphinxparamcomma \sphinxparam{\DUrole{n}{terms}\DUrole{o}{=}\DUrole{default_value}{20}}}
{}
\pysigstopsignatures
\sphinxAtStartPar
Compute sinh(x) using Taylor series expansion.
\begin{quote}\begin{description}
\sphinxlineitem{Parameters}\begin{itemize}
\item {} 
\sphinxAtStartPar
\sphinxstyleliteralstrong{\sphinxupquote{np.array}} (\sphinxstyleliteralemphasis{\sphinxupquote{int}}\sphinxstyleliteralemphasis{\sphinxupquote{ or }}\sphinxstyleliteralemphasis{\sphinxupquote{float}}\sphinxstyleliteralemphasis{\sphinxupquote{ or }}\sphinxstyleliteralemphasis{\sphinxupquote{list}}\sphinxstyleliteralemphasis{\sphinxupquote{ of }}\sphinxstyleliteralemphasis{\sphinxupquote{integers or}}) \textendash{} x: The input value for sinh(x). Can be a scalar or a numpy array.

\item {} 
\sphinxAtStartPar
\sphinxstyleliteralstrong{\sphinxupquote{int}} \textendash{} \begin{description}
\sphinxlineitem{terms: Number of terms in the Taylor series to consider}
\sphinxAtStartPar
(higher = more accurate).

\end{description}


\end{itemize}

\sphinxlineitem{Returns}
\sphinxAtStartPar
Approximate value of sinh(x). Can be a scalar or a numpy array.

\sphinxlineitem{Return type}
\sphinxAtStartPar
np.float or np.array

\end{description}\end{quote}
\subsubsection*{Examples}

\begin{sphinxVerbatim}[commandchars=\\\{\}]
\PYG{g+gp}{\PYGZgt{}\PYGZgt{}\PYGZgt{} }\PYG{n}{sinh}\PYG{p}{(}\PYG{l+m+mi}{5}\PYG{p}{,}\PYG{n}{terms} \PYG{o}{=} \PYG{l+m+mi}{30}\PYG{p}{)}
\PYG{g+go}{array(74.20321058)}
\end{sphinxVerbatim}

\begin{sphinxVerbatim}[commandchars=\\\{\}]
\PYG{g+gp}{\PYGZgt{}\PYGZgt{}\PYGZgt{} }\PYG{n}{sinh}\PYG{p}{(}\PYG{p}{[}\PYG{l+m+mf}{1.0}\PYG{p}{,} \PYG{l+m+mf}{2.0}\PYG{p}{,} \PYG{l+m+mf}{3.0}\PYG{p}{,} \PYG{l+m+mf}{4.0}\PYG{p}{,} \PYG{l+m+mf}{5.0}\PYG{p}{]}\PYG{p}{)}
\PYG{g+go}{array([ 1.17520119,  3.62686041, 10.01787493, 27.2899172 , 74.20321058])}
\end{sphinxVerbatim}

\end{fulllineitems}

\index{tanh() (in module ascefunctions.taylors\_series.trig\_functions)@\spxentry{tanh()}\spxextra{in module ascefunctions.taylors\_series.trig\_functions}}

\begin{fulllineitems}
\phantomsection\label{\detokenize{index:ascefunctions.taylors_series.trig_functions.tanh}}
\pysigstartsignatures
\pysiglinewithargsret
{\sphinxcode{\sphinxupquote{ascefunctions.taylors\_series.trig\_functions.}}\sphinxbfcode{\sphinxupquote{tanh}}}
{\sphinxparam{\DUrole{n}{x}}\sphinxparamcomma \sphinxparam{\DUrole{n}{terms}\DUrole{o}{=}\DUrole{default_value}{20}}}
{}
\pysigstopsignatures
\sphinxAtStartPar
Compute tanh(x) using Taylor series expansion by
dividing sinh(x) by cosh(x).
\begin{quote}\begin{description}
\sphinxlineitem{Parameters}\begin{itemize}
\item {} 
\sphinxAtStartPar
\sphinxstyleliteralstrong{\sphinxupquote{np.array}} (\sphinxstyleliteralemphasis{\sphinxupquote{int}}\sphinxstyleliteralemphasis{\sphinxupquote{ or }}\sphinxstyleliteralemphasis{\sphinxupquote{float}}\sphinxstyleliteralemphasis{\sphinxupquote{ or }}\sphinxstyleliteralemphasis{\sphinxupquote{list}}\sphinxstyleliteralemphasis{\sphinxupquote{ of }}\sphinxstyleliteralemphasis{\sphinxupquote{integers or}}) \textendash{} x: The input value for tanh(x). Can be a scalar or a numpy array.

\item {} 
\sphinxAtStartPar
\sphinxstyleliteralstrong{\sphinxupquote{int}} \textendash{} \begin{description}
\sphinxlineitem{terms: Number of terms in the Taylor series for sinh(x) and cosh(x)}
\sphinxAtStartPar
to consider (higher = more accurate).

\end{description}


\end{itemize}

\sphinxlineitem{Returns}
\sphinxAtStartPar
Approximate value of tanh(x). Can be a scalar or a numpy array.

\sphinxlineitem{Return type}
\sphinxAtStartPar
np.float or np.array

\end{description}\end{quote}
\subsubsection*{Examples}

\begin{sphinxVerbatim}[commandchars=\\\{\}]
\PYG{g+gp}{\PYGZgt{}\PYGZgt{}\PYGZgt{} }\PYG{n}{tanh}\PYG{p}{(}\PYG{l+m+mi}{1}\PYG{p}{)}
\PYG{g+go}{np.float64(0.7615941559557649)}
\end{sphinxVerbatim}

\begin{sphinxVerbatim}[commandchars=\\\{\}]
\PYG{g+gp}{\PYGZgt{}\PYGZgt{}\PYGZgt{} }\PYG{n}{tanh}\PYG{p}{(}\PYG{p}{[}\PYG{l+m+mf}{1.0}\PYG{p}{,} \PYG{l+m+mf}{2.0}\PYG{p}{,} \PYG{l+m+mf}{3.0}\PYG{p}{,} \PYG{l+m+mf}{4.0}\PYG{p}{,} \PYG{l+m+mf}{5.0}\PYG{p}{]}\PYG{p}{)}
\PYG{g+go}{array([0.76159416, 0.96402758, 0.99505475, 0.9993293 , 0.9999092 ])}
\end{sphinxVerbatim}

\end{fulllineitems}

\subsubsection*{References}


\section{\sphinxstylestrong{Here are the functions in the bessel\_group module:}}
\label{\detokenize{index:module-ascefunctions.bessel_group.bessel_function}}\label{\detokenize{index:here-are-the-functions-in-the-bessel-group-module}}\index{module@\spxentry{module}!ascefunctions.bessel\_group.bessel\_function@\spxentry{ascefunctions.bessel\_group.bessel\_function}}\index{ascefunctions.bessel\_group.bessel\_function@\spxentry{ascefunctions.bessel\_group.bessel\_function}!module@\spxentry{module}}\index{bessel() (in module ascefunctions.bessel\_group.bessel\_function)@\spxentry{bessel()}\spxextra{in module ascefunctions.bessel\_group.bessel\_function}}

\begin{fulllineitems}
\phantomsection\label{\detokenize{index:ascefunctions.bessel_group.bessel_function.bessel}}
\pysigstartsignatures
\pysiglinewithargsret
{\sphinxcode{\sphinxupquote{ascefunctions.bessel\_group.bessel\_function.}}\sphinxbfcode{\sphinxupquote{bessel}}}
{\sphinxparam{\DUrole{n}{alpha}}\sphinxparamcomma \sphinxparam{\DUrole{n}{x}}\sphinxparamcomma \sphinxparam{\DUrole{n}{terms}\DUrole{o}{=}\DUrole{default_value}{50}}}
{}
\pysigstopsignatures
\sphinxAtStartPar
Compute the bessel function for a scalar or numpy array.
\begin{quote}\begin{description}
\sphinxlineitem{Parameters}\begin{itemize}
\item {} 
\sphinxAtStartPar
\sphinxstyleliteralstrong{\sphinxupquote{np.array}} (\sphinxstyleliteralemphasis{\sphinxupquote{int}}\sphinxstyleliteralemphasis{\sphinxupquote{ or }}\sphinxstyleliteralemphasis{\sphinxupquote{float}}\sphinxstyleliteralemphasis{\sphinxupquote{ or }}\sphinxstyleliteralemphasis{\sphinxupquote{list}}\sphinxstyleliteralemphasis{\sphinxupquote{ of }}\sphinxstyleliteralemphasis{\sphinxupquote{integers or}}) \textendash{} alpha: Order of the Bessel function.

\item {} 
\sphinxAtStartPar
\sphinxstyleliteralstrong{\sphinxupquote{np.array}} \textendash{} x: Input value(s) at which to evaluate the Bessel function.

\end{itemize}

\sphinxlineitem{Returns}
\sphinxAtStartPar
Approximation of the Bessel function.

\sphinxlineitem{Return type}
\sphinxAtStartPar
np.float or np.array

\end{description}\end{quote}
\subsubsection*{Example}

\begin{sphinxVerbatim}[commandchars=\\\{\}]
\PYG{g+gp}{\PYGZgt{}\PYGZgt{}\PYGZgt{} }\PYG{n}{bessel\PYGZus{}function}\PYG{p}{(}\PYG{l+m+mi}{1}\PYG{p}{,}\PYG{p}{[}\PYG{l+m+mi}{0}\PYG{p}{,} \PYG{l+m+mi}{1}\PYG{p}{,} \PYG{l+m+mi}{2}\PYG{p}{,} \PYG{l+m+mi}{3}\PYG{p}{,} \PYG{l+m+mi}{4}\PYG{p}{,} \PYG{l+m+mi}{5}\PYG{p}{]}\PYG{p}{)}
\PYG{g+go}{array([ 0.        ,  0.4404682 ,  0.57755987,  0.34031109, \PYGZhy{}0.06437481,}
\PYG{g+go}{       \PYGZhy{}0.32549533])}
\end{sphinxVerbatim}

\end{fulllineitems}

\index{factorial() (in module ascefunctions.bessel\_group.bessel\_function)@\spxentry{factorial()}\spxextra{in module ascefunctions.bessel\_group.bessel\_function}}

\begin{fulllineitems}
\phantomsection\label{\detokenize{index:ascefunctions.bessel_group.bessel_function.factorial}}
\pysigstartsignatures
\pysiglinewithargsret
{\sphinxcode{\sphinxupquote{ascefunctions.bessel\_group.bessel\_function.}}\sphinxbfcode{\sphinxupquote{factorial}}}
{\sphinxparam{\DUrole{n}{n}}}
{}
\pysigstopsignatures
\sphinxAtStartPar
Compute factorial for a scalar or numpy array.
:param int or float or list of integers or np.array: n: Input value(s) for which to compute the factorial.
\begin{quote}\begin{description}
\sphinxlineitem{Returns}
\sphinxAtStartPar
Factorial of the input(s).

\sphinxlineitem{Return type}
\sphinxAtStartPar
np.float or np.array

\end{description}\end{quote}
\subsubsection*{Example}

\begin{sphinxVerbatim}[commandchars=\\\{\}]
\PYG{g+gp}{\PYGZgt{}\PYGZgt{}\PYGZgt{} }\PYG{n}{factorial}\PYG{p}{(}\PYG{p}{[}\PYG{l+m+mi}{0}\PYG{p}{,} \PYG{l+m+mi}{1}\PYG{p}{,} \PYG{l+m+mi}{2}\PYG{p}{,} \PYG{l+m+mi}{3}\PYG{p}{,} \PYG{l+m+mi}{4}\PYG{p}{,} \PYG{l+m+mi}{5}\PYG{p}{]}\PYG{p}{)}
\PYG{g+go}{array([  1.,   1.,   2.,   6.,  24., 120.])}
\end{sphinxVerbatim}

\end{fulllineitems}

\index{gamma\_function() (in module ascefunctions.bessel\_group.bessel\_function)@\spxentry{gamma\_function()}\spxextra{in module ascefunctions.bessel\_group.bessel\_function}}

\begin{fulllineitems}
\phantomsection\label{\detokenize{index:ascefunctions.bessel_group.bessel_function.gamma_function}}
\pysigstartsignatures
\pysiglinewithargsret
{\sphinxcode{\sphinxupquote{ascefunctions.bessel\_group.bessel\_function.}}\sphinxbfcode{\sphinxupquote{gamma\_function}}}
{\sphinxparam{\DUrole{n}{z}}}
{}
\pysigstopsignatures
\sphinxAtStartPar
Compute Gamma function for a scalar or numpy array.
\begin{quote}

\sphinxAtStartPar
Parameters
\end{quote}
\begin{quote}\begin{description}
\sphinxlineitem{Returns}
\sphinxAtStartPar
Gamma function value of the input(s).

\sphinxlineitem{Return type}
\sphinxAtStartPar
np.float or np.array

\end{description}\end{quote}
\subsubsection*{Example}

\begin{sphinxVerbatim}[commandchars=\\\{\}]
\PYG{g+gp}{\PYGZgt{}\PYGZgt{}\PYGZgt{} }\PYG{n}{gamma\PYGZus{}function}\PYG{p}{(}\PYG{p}{[}\PYG{l+m+mi}{1}\PYG{p}{,} \PYG{l+m+mf}{1.5}\PYG{p}{,} \PYG{l+m+mi}{2}\PYG{p}{,} \PYG{l+m+mf}{2.5}\PYG{p}{,} \PYG{l+m+mi}{3}\PYG{p}{,} \PYG{l+m+mi}{4}\PYG{p}{]}\PYG{p}{)}
\PYG{g+go}{array([1.05088491, 0.87972523, 0.99916542, 1.32925696, 1.99999916,}
\PYG{g+go}{       6.00000083])}
\end{sphinxVerbatim}

\end{fulllineitems}



\renewcommand{\indexname}{Python Module Index}
\begin{sphinxtheindex}
\let\bigletter\sphinxstyleindexlettergroup
\bigletter{a}
\item\relax\sphinxstyleindexentry{ascefunctions.bessel\_group.bessel\_function}\sphinxstyleindexpageref{index:\detokenize{module-ascefunctions.bessel_group.bessel_function}}
\item\relax\sphinxstyleindexentry{ascefunctions.taylors\_series}\sphinxstyleindexpageref{index:\detokenize{module-ascefunctions.taylors_series}}
\item\relax\sphinxstyleindexentry{ascefunctions.taylors\_series.trig\_functions}\sphinxstyleindexpageref{index:\detokenize{module-ascefunctions.taylors_series.trig_functions}}
\end{sphinxtheindex}

\renewcommand{\indexname}{Index}
\printindex
\end{document}